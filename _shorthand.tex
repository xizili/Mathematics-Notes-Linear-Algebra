%% Rather hacky definition of a plain remark/note by riding on \added
\newcommand{\note}[2][]{\added[#1,comment={#2}]{}}
\newcommand{\annotate}{\added}

\newtheorem{proposition}{Proposition}[section]
\newtheorem{theorem}[proposition]{Theorem}
\newtheorem{example}[proposition]{Example}
\newtheorem{corollary}[proposition]{Corollary}
\newtheorem{definition}[proposition]{Definition}
\newtheorem{remark}[proposition]{Remark}
\newtheorem{lemma}[proposition]{Lemma}
\newtheorem{warning}[proposition]{Warning}
\newtheorem{notation}[proposition]{Notation}


\usepackage{tcolorbox}
\tcbuselibrary{theorems}
\tcbuselibrary{breakable}
\newtcbtheorem[number within=]{textremark}{Remark}%
{colback=green!5,colframe=green!35!black,fonttitle=\bfseries,breakable}{th}
\creflabelformat{equation}{#2#1#3}


\renewcommand\b\mathbf
\newcommand{\T}{\mathrm{T}}
\newcommand{\f}{\operatorname{f}}
\newcommand{\g}{\operatorname{g}}
\newcommand{\VF}{\mathcal{V}_{\mathcal{F}}}
\newcommand{\VR}{\mathcal{V}_{\mathbb{R}}}
\newcommand{\VC}{\mathcal{V}_{\mathbb{C}}}
\newcommand{\F}{{\mathcal{F}}}
\newcommand{\G}{\operatorname{G}}
\newcommand{\R}{\mathbb{R}}
\newcommand{\C}{\mathbb{C}}
\newcommand{\x}{\mathbf{x}}
\newcommand{\y}{\mathbf{y}}
\newcommand{\vv}{\mathbf{v}}
\newcommand{\w}{\mathbf{w}}
\newcommand{\e}{\mathbf{e}}
\newcommand{\WF}{\mathcal{W}_{\mathcal{F}}}
\newcommand{\V}{\mathcal{V}}
\newcommand{\W}{\mathcal{W}}

