\section{Hi! Jordan Canonical Form}

Jordan canonical form is the most insightful theorem in linear algebra that reveals the relationship of the most essential concepts. 
To understand it will help us get clear about the underlying links among concepts such as eigenvalues, linearity, change-of-basis, etc.


% state JCF

\begin{theorem} \label{JCF}
\textbf{(Jordan Canonical Form)} For any linear map $\f : \VF \to \VF$ ($\mathcal{V}$ is vector space of finite dimensions with field $\F$) and a chosen basis, there exists a unique upper triangular matrix $\b{J}$  of the form:\\
$$\begin{bmatrix}
    \begin{bmatrix}
        \lambda_1 & 1 & \cdots & 0 &0 \\
        0 &\lambda_1 & \cdots &0 &0\\
        \vdots &\vdots  &\ddots &\vdots &\vdots \\
        0 &0 &\cdots &\lambda_1 &1 \\
        0 &0 &\cdots &0 &\lambda_1
    \end{bmatrix}
    &\b{0} & \cdots &\b{0} \\
    \b{0}   &\begin{bmatrix}
        \lambda_2 & 1 & \cdots &0 &0 \\
        0 &\lambda_2 & \cdots &0 &0\\
        \vdots &\vdots  &\ddots &\vdots &\vdots \\
        0 &0 &\cdots &\lambda_2 &1 \\
        0 &0 &\cdots &0 &\lambda_2
        \end{bmatrix}
    & \cdots &\b{0}  \\
    \vdots &\vdots &\ddots &\vdots \\
    \b{0}  &\b{0}  &\cdots &\begin{bmatrix}
                    \lambda_n & 1 & \cdots &0 &0 \\
                     0 &\lambda_n & \cdots &0 &0\\
                     \vdots &\vdots  &\ddots &\vdots &\vdots \\
                    0 &0 &\cdots &\lambda_n &1 \\
                    0 &0 &\cdots &0 &\lambda_n
                    \end{bmatrix}
\end{bmatrix}
\begin{array}[c]{@{}l@{\,}l}
    \left.\begin{array}{c} 
    \vphantom{0}  \\ \vphantom{0} \\ \vphantom{\vdots} \\ \vphantom{0} \\ \vphantom{0} 
    \end{array} \right\} & \text{$k_1$ times} \\
    \left. \begin{array}{c}
        \vphantom{0}  \\ \vphantom{0} \\ \vphantom{\vdots} \\ \vphantom{0} \\ \vphantom{0}
    \end{array} \right\} & \text{$k_2$ times} \\
    \vphantom{\vdots} \\
    \left. \begin{array}{c} 
        \vphantom{0}  \\ \vphantom{0} \\ \vphantom{\vdots} \\ \vphantom{0} \\ \vphantom{0}
    \end{array} \right\} & \text{$k_n$ times}
\end{array}
$$
where integers $k_i \geq 1$ and $\lambda_i$ are not necessarily distinct for $i \in \{1,2, \cdots, n\}$, such that $\f(\vv) = \b{J}\vv$ for $\forall \vv \in \VF$.
Matrix $\b{J}$, as the \textbf{matrix representation} of the linear map $\f$ with respect to the chosen basis,  is called a \textbf{Jordan matrix} or \textbf{Jordan canonical form}.
Upper triangular submatrices along diagonal are called \textbf{Jordan blocks}.
\end{theorem}

\begin{remark}
$k_i$ is equal to the algebraic multiplicity of the root $\lambda_i$ in the characteristic polynomial of $f$.
\end{remark}




\subsection{The logic of proof}

% how to prove JCF
\begin{theorem}
For a linear map $\f: \VF \to \VF$, if $S_1$ and $S_2$ are subspaces of $\VF$ and they satisfy \\
$~\ \, \bullet$ 
%
\annotate[id=r, comment={
It reads "V is generated by two orthogonal subspaces".\\
This condition is equivalent to \\
$S_1 \cap S_2 = \{0\} \ \& \  S_1 + S_2 = V$. \\ 
The latter half reads ``$S_1$ and $S_2$ generate $V$" and is defined as below: \\
for $\forall v \in V, \exists s_1 \in S_1, s_2 \in S_2, \textnormal{s.t.}, s_1 + s_2 = v$.
}]
{$S_1 \oplus S_2 = V$}; \\
%
$~\ \, \bullet$
%
\annotate[id=r, comment={This condition means \\ $f({S_1}) \subseteq S_1 \ \& \ f({S_2}) \subseteq S_2$.}]
{$f$ preserves $S_1$ and $S_2$}; \\
%
$~\ \, \bullet$ $f: S_1 \to S_1$ (abbr. $f|_{S_1}$) has a matrix representation $\b{M}_1$ and $f: S_2 \to S_2$ has a matrix representation $\b{M}_2$;\\
then $f|_{V}$ has a matrix representation of the form
$$
\begin{bmatrix}
\b{M}_1 & \b{0} \\
\b{0} & \b{M}_2
\end{bmatrix}.
$$
\end{theorem}
%

\begin{corollary} \label{JCF_prooflogic}
For a linear map $f: V \to V$, if $S_i$ is a subspace of $V$ with $i \in \{1, 2, \cdots, n\}$ and $S_i$ satisfies \\
$~\ \, \bullet$ \annotate[id=r, comment={
This condition is equivalent to \\
$S_i \cap \sum_{j \ne \ i} S_j = \{0\}$\\ 
$\& \ \sum_{i} S_i = V.$ \\
Note that the former half is not\\
$S_i \cap  S_j = \{0\}$.
}]
{$\underset{i}{\oplus} S_i = V$}; \\
%
$~\ \, \bullet$ $f$ preserves $S_i$; \\
%
$~\ \, \bullet$  $f|_{S_i}$ has a matrix representation $\b{M}_i$;\\
then $f|_{V}$ has a matrix representation of the form
$$
\begin{bmatrix}
\b{M}_1 & \b{0}  &\cdots & \b{0} \\
\b{0}   &\b{M}_2 &\cdots & \b{0} \\
\vdots  &\vdots  &\cdots & \vdots \\
\b{0}   &\b{0}   &\cdots & \b{M}_n
\end{bmatrix}.
$$
\end{corollary}

\newpage

\noindent Following \cref{JCF_prooflogic}, we propose a way of proving Jordan Canonical theorem (\cref{JCF}) as below.
Let $S_i$ be generalised eigenspaces of the vector space $V$ associated with the linear map $f$.
Jordan Canonical theorem follows, if we can prove that \\
$~\ \qquad \bullet$ $\underset{i}{\oplus} S_i = V$;\\
$~\ \qquad \bullet$ $f$ preserves $S_i$; \\
$~\ \qquad \bullet$ $f|_{S_i}$ has a matrix representation in the form of accumulated Jordan blocks.

\noindent We will show it step-by-step below.


\begin{definition}[\textbf{generalised eigenspace}]
Let an identity map be $\operatorname{e}:\VF to \VF$.
For a linear map $\f$, if $[(\f - \lambda \operatorname{e})^k] (\vv) = \b{0}$ for some positive integer $k$ and some $\lambda \in \F$, 
Generalised eigenspaces
\end{definition}

\begin{theorem}
Let $S_i$ be generalised eigenspaces of the vector space $V$ associated with the linear map $\f: \VF \to \VF$. 
Then, $f$ preserves $S_i$ and $\underset{i}{\oplus} S_i = V$.
\end{theorem}









$$
\begin{bmatrix}
    \begin{bmatrix}
        \lambda & 1 & \cdots & 0 &0 \\
        0 &\lambda & \cdots &0 &0\\
        \vdots &\vdots  &\ddots &\vdots &\vdots \\
        0 &0 &\cdots &\lambda &1 \\
        0 &0 &\cdots &0 &\lambda
    \end{bmatrix}
    &0 & \cdots &0 \\
    0  &\begin{bmatrix}
    \lambda & 1 & \cdots &0 &0 \\
    0 &\lambda & \cdots &0 &0\\
    \vdots &\vdots  &\ddots &\vdots &\vdots \\
    0 &0 &\cdots &\lambda &1 \\
    0 &0 &\cdots &0 &\lambda
    \end{bmatrix}
    & \cdots &0 \\
    \vdots &\vdots &\ddots &\vdots \\
    0 &0 &\cdots &\begin{bmatrix}
                    \lambda & 1 & \cdots &0 &0 \\
                     0 &\lambda & \cdots &0 &0\\
                     \vdots &\vdots  &\ddots &\vdots &\vdots \\
                    0 &0 &\cdots &\lambda &1 \\
                    0 &0 &\cdots &0 &\lambda
                    \end{bmatrix}
\end{bmatrix}
\begin{array}[c]{@{}l@{\,}l}
    \left.\begin{array}{c} 
    \vphantom{0}  \\ \vphantom{0} \\ \vphantom{\vdots} \\ \vphantom{0} \\ \vphantom{0} 
    \end{array} \right\} & \text{$l_1$ times} \\
    \left. \begin{array}{c}
        \vphantom{0}  \\ \vphantom{0} \\ \vphantom{\vdots} \\ \vphantom{0} \\ \vphantom{0}
    \end{array} \right\} & \text{$l_2$ times} \\
    \vphantom{\vdots} \\
    \left. \begin{array}{c} 
        \vphantom{0}  \\ \vphantom{0} \\ \vphantom{\vdots} \\ \vphantom{0} \\ \vphantom{0}
    \end{array} \right\} & \text{$l_m$ times}
\end{array}
$$
where the integers $l_i \geq 1$ for $i \in \{1,2, \cdots, m\}$.
An example matrix of such kind could be
$$
\begin{bmatrix}
    \begin{bmatrix}
        \lambda & 1  & 0 &0 \\
        0 &\lambda  &1 &0\\
        0 &0 &\lambda &1 \\
        0 &0 &0 &\lambda
    \end{bmatrix}
    &0 &0 &0 \\
    0  &\begin{bmatrix}
        \lambda & 1\\
        0 &\lambda
        \end{bmatrix}
    &0 &0 \\
    0  &0 &\begin{bmatrix}
        \lambda & 1\\
        0 &\lambda
        \end{bmatrix}
    &0 \\
    0 &0 &0 &\lambda
\end{bmatrix}.
$$

% Lorem ipsum \annotate[id=r,comment={$ay=p$}]{dolor sit amet}, consectetur adipiscing elit. Integer luctus molestie hendrerit. Suspendisse nec tellus tellus.\note[id=b]{$xx=yy$} \deleted[id=r]{aaaaaaa}{bbbbbbbbbb}

% luctus et ultrices posuere cubilia Curae; \replaced[id=r]{Donec et massa in arcu sagittis posuere a a erat.}{Vivamus tincidunt nec purus vitae porttitor.} umst. Aliquam molestie, sem eu suscipit pharetra, odio turpis tristique odio, molestie egestas sapien massa in elit.\note[id=b]{Fusce eget sagittis eros. Donec a posuere elit, vitae congue augue.} Donec semper mi felis, id cursus turpis pretium in.